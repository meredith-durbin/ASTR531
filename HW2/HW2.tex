\documentclass[11pt]{article}
\usepackage[margin=1in, lmargin=0.75in]{geometry}
\usepackage{caption}
\usepackage{float}
\usepackage{graphicx}
\usepackage{latexsym}
\usepackage{amsmath}
\usepackage{cancel}
\usepackage{astro}
\usepackage{url}
\usepackage[bottom]{footmisc}

\newcommand\allbold[1]{{\boldmath\textbf{#1}}}
\newcommand\pc{\mathrm{\ pc}}
\newcommand\Lpc{\ L_\odot/\!\!\pc^2}
\newcommand\sech{\ \!\mathrm{sech}}
\newcommand\lsol{\mathrm{L}_\odot}
\newcommand\rsol{\mathrm{R}_\odot}
\newcommand\msol{\mathrm{M}_\odot}

\begin{document}

\begin{flushright}Meredith Durbin\\
Emily Levesque\\
Astro 531: Stellar Interiors\\
\today\\

\end{flushright}

\center{\textsc{Homework 2}} \\[6pt]

All calculations can be found in the notebook \url{https://github.com/meredith-durbin/ASTR531/blob/master/HW2/HW2.ipynb}.

\begin{enumerate}

\item [8.2]
	\begin{enumerate}
	
    \item 
    
    \end{enumerate}

\item [9.1] Timescales for various stars:

    \begin{table}[H]
    \centering
    \begin{tabular}{ r | c | c | c }
      Star & $\tau_\mathrm{dyn}$ & $\tau_\mathrm{KH}$ & $\tau_\mathrm{nucl}$ \\ \hline
      MS, $1~\msol$ & 0.906 h & $3.140\times10^7$ yr & $10^{10}$ yr  \\
      MS, $60~\msol$ & 6.792 h & $9.487\times10^3$ yr & $7.554\times10^{5}$ yr  \\
      RSG, $15~\msol$ & 5.056 yr & 4.793 yr & $3.358\times10^{5}$ yr   \\
      WD, $0.6~\msol$ & 7.142 s & $7.945\times10^{10}$ yr & ---  \\
    \end{tabular}
    \end{table}

\item [9.2]
    If nuclear fusion in the sun were to suddenly stop, it would take approximately a thermal timescale to notice.

\item [11.2]
	\begin{enumerate}
	
    \item 
    
    \end{enumerate}

\item [12.2] % for 0.1, 1, 10, and 100Mo stars
	\begin{enumerate}
    
    \item 
    	\begin{table}[H]
        \centering
        \begin{tabular}{ r | c | c | c | c }
          Mass  &  &  &  &  \\ \hline
          $0.1~\msol$ &  &  &  &  \\
          $1~\msol$ &  &  &  &  \\
          $10~\msol$ &  &  &  &  \\
          $100~\msol$ &  &  &  &  \\
        \end{tabular}
        \end{table}

        
    \end{enumerate}

\item [13.2]
	For X = Y = 0.49, the luminosity, radius, and $T_\mathrm{eff}$ are respectively 227.8\%, 111.5\%, and 123.3\% of the corresponding quantities at solar abundances.


\end{enumerate}
\end{document}