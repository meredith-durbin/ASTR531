\documentclass[11pt]{article}
\usepackage[margin=1in, lmargin=0.75in]{geometry}
\usepackage{caption}
\usepackage{float}
\usepackage{graphicx}
\usepackage{latexsym}
\usepackage{amsmath}
\usepackage{cancel}
\usepackage{astro}
\usepackage{url}
\usepackage[bottom]{footmisc}

\newcommand\allbold[1]{{\boldmath\textbf{#1}}}
\newcommand\pc{\mathrm{\ pc}}
\newcommand\Lpc{\ L_\odot/\!\!\pc^2}
\newcommand\sech{\ \!\mathrm{sech}}
\newcommand\lsol{\mathrm{L}_\odot}
\newcommand\rsol{\mathrm{R}_\odot}
\newcommand\msol{\mathrm{M}_\odot}

\begin{document}

\begin{flushright}Meredith Durbin\\
Emily Levesque\\
Astro 531: Stellar Interiors\\
\today\\

\end{flushright}

\center{\textsc{Homework 2}} \\[6pt]

All calculations can be found in the notebook \url{https://github.com/meredith-durbin/ASTR531/blob/master/HW2/HW2.ipynb}.

\begin{enumerate}

\item [8.2]
	\begin{enumerate}
	
    \item The helium-burning core must be less massive than the hydrogen-burning core for a given stellar mass; for a given temperature/density profile, less of the enclosed mass will be at the temperature required to kick off helium vs. hydrogen burning.
    
    \item The star will be more chemically stratified at the end of the helium-burning phase, with the helium fusion products at the core and a shell of unfused helium surrounding them.
    
    \end{enumerate}

\item [9.1] Timescales for various stars:

    \begin{table}[H]
    \centering
    \begin{tabular}{ r | c | c | c }
      Star & $\tau_\mathrm{dyn}$ & $\tau_\mathrm{KH}$ & $\tau_\mathrm{nucl}$ \\ \hline
      MS, $1~\msol$ & 0.906 h & $3.140\times10^7$ yr & $10^{10}$ yr  \\
      MS, $60~\msol$ & 6.792 h & $9.487\times10^3$ yr & $7.554\times10^{5}$ yr  \\
      RSG, $15~\msol$ & 5.056 yr & 4.793 yr & $3.358\times10^{5}$ yr   \\
      WD, $0.6~\msol$ & 7.142 s & $7.945\times10^{10}$ yr & ---  \\
    \end{tabular}
    \end{table}

\item [9.2]
    If nuclear fusion in the sun were to suddenly stop, it would take approximately a thermal timescale to notice; the solar spectrum we observe is largely a product of temperature, and the thermal timescale is the timescale over which a change in temperature would become noticeable.

\item [11.2]
	\begin{enumerate}
	
    \item For $1~\msol$, I find $\beta = 0.9996$ and $P_\mathrm{rad}/P_\mathrm{gas} = 0.0004$, and for $60~\msol$, I find $\beta = 0.6867$ and $P_\mathrm{rad}/P_\mathrm{gas} = 0.4562$.
    
    \item For $1~\msol$ I find a predicted luminosity $L = L_\mathrm{Edd}(1-\beta) = 14.8~\lsol$, and for $60~\msol$ I find $L = 7.14\times10^5~\lsol$.
    
    \item My luminosity is off by a factor of $\sim\!20$ for $1~\msol$, but is close for $60~\msol$.
    
    \end{enumerate}

\item [12.2] % for 0.1, 1, 10, and 100Mo stars
	\begin{enumerate}
    
    \item All radii are in solar radii.
    	\begin{table}[H]
        \centering
        \begin{tabular}{ r | c | c | c }
          Mass ($\msol$) & $R_\mathrm{start,H}$ & $R_\mathrm{end,H}$ & $R_\mathrm{end,PMS}$ \\ \hline
          $0.1$ & 10 & 0.2 & 0.2 \\
          $1$ & 100 & 2 & 1 \\
          $10$ & 1000 & 20 & 5 \\
          $100$ & 10$^4$ & 200 & 25 \\
        \end{tabular}
        \end{table}

	\item 
        Timescales:
    	\begin{table}[H]
        \centering
        \begin{tabular}{ r | c | c }
          Mass ($\msol$)  & $\tau_\mathrm{Hayashi}$ (yr) & $\tau_\mathrm{PMS}$ (yr) \\ \hline
          $0.1$ & $10^7$ & $1.897\times10^{10}$  \\
          $1$ & $10^6$ & $6\times10^{7}$  \\
          $10$ & $10^5$ & $1.897\times10^{5}$  \\
          $100$ & $10^4$ & $6\times10^{2}$ \\
        \end{tabular}
        \end{table}
    \end{enumerate}

\item [13.2]
	I chose to calculate the ratio of final to initial quantities so that I could ignore mass entirely.
	\begin{align}
	\mu_0 &= (2-1.25Y_0)^{-1} \\
	\mu_1 &= (2-1.25Y_1)^{-1} \\
	L_1/L_0 &= \frac{ (2-Y_1)^{-1} \mu_1^4 }{ (2-Y_0)^{-1} \mu_1^4 } \\
	R_1/R_0 &= \frac{ (1+X_1)^{0.05} \mu_1^{2/3} }{ (1+X_0)^{0.05} \mu_0^{2/3} } \\
	T_1/T_0 &= \frac{ (1+X_1)^{-0.5} \mu_1^{0.83} }{ (1+X_0)^{-0.5} \mu_0^{0.83} }
	\end{align}
	For $X_1 = Y_1 = 0.49$ and $X_0 = 0.7$, $Y_0 = 0.28$, $L_1/L_0 = 2.28$, $R_1/R_0 = 1.12$, and $T_1/T_0 = 1.23$.


\end{enumerate}
\end{document}