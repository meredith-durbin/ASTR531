\documentclass[11pt]{article}
\usepackage[margin=1in, lmargin=0.75in]{geometry}
\usepackage{caption}
\usepackage{float}
\usepackage{graphicx}
\usepackage{latexsym}
\usepackage{amsmath}
\usepackage{cancel}
\usepackage{astro}
\usepackage{url}
\usepackage[bottom]{footmisc}

\newcommand\allbold[1]{{\boldmath\textbf{#1}}}
\newcommand\pc{\mathrm{\ pc}}
\newcommand\Lpc{\ L_\odot/\!\!\pc^2}
\newcommand\sech{\ \!\mathrm{sech}}
\newcommand\lsol{\mathrm{L}_\odot}
\newcommand\rsol{\mathrm{R}_\odot}
\newcommand\msol{\mathrm{M}_\odot}

\begin{document}

\begin{flushright}Meredith Durbin\\
Emily Levesque\\
Astro 531: Stellar Interiors\\
\today\\

\end{flushright}

\center{\textsc{Homework 5}} \\[6pt]

All calculations can be found in the notebook \url{https://github.com/meredith-durbin/ASTR531/blob/master/HW5/HW5.ipynb}.

\begin{enumerate}

\item [25.4] I don't see how to do this one at all, sorry!
%	\begin{enumerate}
%	
%    \item 
%    
%    \end{enumerate}

\item [26.2] 
	Stars with $M_i > 20~\msol$ are primarily C and O because by the end of their lives they have lost their hydrogen envelope, and the timescales of fusion phases after helium fusion are so short that they do not produce significant amounts of heavier elements relative to C and O.

\item [27.2]
	Type I supernova progenitors are collapsing Wolf-Rayet stars, whereas Type II supernova progenitors are RSGs or LBVs. The main difference between these progenitors is that Wolf-Rayet stars have undergone enough mass loss to be stripped of their hydrogen envelope. Mass loss in hot stars increases with increasing metallicity because of line-driven winds, so at higher metallicity one would expect relatively more Type I supernovae. Rotation also increases the relative number of Type I supernovae, as rotation also increases the mass loss rate.
    
\item [28.1]
	\begin{enumerate}
	
    \item The initial separation is $40.3~\rsol$.
    
    \item The Roche lobe radius for a $25~\msol$ star in a binary with a $10~\msol$ is $18.5~\rsol$ (equation 28.11). The TAMS radius for a $25~\msol$ star at solar metallicity is $18.1~\rsol$, so at the TAMS the star is very close to filling its Roche lobe. 
    
    \item The minimum period and separation are 2.7 days and $26.8~\rsol$.
    
    \item If $15~\msol$ is transferred from the donor to the accretor, the system will return to its initial separation and period.
    
    \end{enumerate}

\end{enumerate}
\end{document}